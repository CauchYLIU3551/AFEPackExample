\documentclass[12pt,openany]{article}
\usepackage[dvips]{graphicx}
\usepackage{latexsym}
\usepackage{epsfig}
\usepackage{color}
\usepackage{amsmath,amsthm,amssymb}
\usepackage{mathrsfs}
\usepackage{array}
\usepackage{graphicx}
\usepackage{epsfig}
\usepackage{cite}
\usepackage{fancyhdr}
\usepackage{listings}

\lstset{language=C++}
\lstset{breaklines}
\lstset{extendedchars=false,
	basicstyle=\footnotesize,
	frame=single,
	tabsize=4
}

\marginparwidth=45pt \topmargin=0.3in \textwidth=5.9in
\oddsidemargin=0.2in \evensidemargin=0.2in \baselineskip=30pt
\parskip=0.1in
\pagenumbering{roman}
\renewcommand{\baselinestretch}{1.2}
\newcommand{\IR}{{\mathbb{R}}}
\newcommand{\IN}{{\mathbb{N}}}
\newcommand{\bff}{{\bf f}}
\newcommand{\bc}{{\bf c}}
\newcommand{\bq}{{\bf q}}
\newcommand{\bp}{{\bf p}}
\newcommand{\bv}{{\bf v}}
\newcommand{\bw}{{\bf w}}
\newcommand{\by}{{\bf y}}
\newcommand{\bx}{{\bf x}}
\newcommand{\BE}{\begin{equation}}
\newcommand{\EE}{\end{equation}}
\newcommand{\non}{\bf}
\newcommand{\bd}{\boldsymbol}

\newtheorem{example}{Example}
\newtheorem{theorem}{Theorem}
\newtheorem{lemma}{Lemma}
\newtheorem{corollary}{Corollary}
\newtheorem{definition}{Definition}
\newtheorem{Remark}{\indent {Remark}}


\def\dfrac{\displaystyle\frac}
\def\dsum{\displaystyle\sum}
\def\dint{\displaystyle\int}

\makeatletter
\newcommand{\rmnum}[1]{\romannumeral #1}
\newcommand{\Rmnum}[1]{\expandafter\@slowromancap\romannumeral #1@}
\makeatother

\title{AFEPack Notes}
\author{Liu Chengyu}
\begin{document}

 %--------------Ŀ¼------------------
\maketitle

\newpage
\pagenumbering{arabic}
\setcounter{page}{1}
%...................question 1...................
\section*{Parabolic Equations}
Finishing the study the example of poisson-equation, I tried to add the $\frac{\partial u}{\partial t}$ into the problem. Thus the problem changed as follow:\\
\begin{align}
	\frac{\partial u}{\partial t}-\bigtriangleup u &= f\ in\   \  \Omega,\\
	u_{t=0}(x,y) &= sin(\pi(x))sin(2\pi y),\\
	u &= u\   on\   \partial \Omega,\\
	\Omega&=[0,1]x[0,1].
\end{align}

In fact, we can set the $u=sin(\pi(x+t))sin(2\pi y)$, then we can get the exact f as follow:
 $$f=\pi cos(\pi(x+t))sin(2\pi y)+5\pi^{2}sin(\pi(x+t))sin(2\pi y)$$
Due to the special form of the function $u$ in (1), we could image that the graph of new equation would show as the graph of the original poisson equation moving through the x-axis.\\
 Then I use the AFEPack to solve the problem to verify our assumption. At first, I discretized the equation in time as follow:
\begin{align}
	\frac{u^{n}-u^{n-1}}{dt} - \bigtriangleup u^{n} = f .
\end{align}
then I can transform the (5) as :
\begin{align}
	\frac{u^{n}}{dt}-\bigtriangleup u^{n} = f + \frac{u^{n-1}}{dt}.
\end{align}
Multiplying trial functions v in both side of (6), then we can define the new bilinear operater $a(u,v)$ as :
\begin{align}
	\int_{\Omega}(\frac{u^{n}}{dt}v+\bigtriangledown u\cdot\bigtriangledown v) = \int_{\Omega}(f+\frac{u^{n-1}}{dt}).
\end{align}
In fact, we have the result of $u^{n-1}$ at n time step and we just need to calculate the value of the $u^{n}$. So the (7) can be converted as:
\begin{align}
	\int_{\Omega}\frac{uv}{dt}+\bigtriangledown u\cdot\bigtriangledown v = \int_{\Omega}f+\frac{u_{h}}{dt}.
\end{align}
in which $u_h$ represents the result at n-1 time step. Obviously, (8) is different as the Galerkin form of poisson equation. So I change the function getElementmatrix in the .cpp file to build a new stiff matrix. What's more, I also add the new item $\frac{u_{h}}{dt}$ in to the right hand side as follow codes:

\begin{lstlisting}
class Matrix : public L2InnerProduct<DIM,double>
{
	private:
	double _dt;
	public:
	Matrix(FEMSpace<double,DIM>& sp, double dt) :
	L2InnerProduct<DIM,double>(sp, sp), _dt(dt) {}
	virtual void getElementMatrix(const Element<double,DIM>& e0,
	const Element<double,DIM>& e1,
	const ActiveElementPairIterator< DIM >::State s)
	{
		double vol = e0.templateElement().volume();
		u_int acc = algebricAccuracy();
		...
		...
		for (u_int l = 0;l < n_q_pnt;++ l) {
			double Jxw = vol*qi.weight(l)*jac[l];
			for (u_int i = 0;i < n_ele_dof;++ i) {
				for (u_int j = 0;j < n_ele_dof;++ j) {
					elementMatrix(i,j) += Jxw*(bas_val[i][l]*bas_val[j][l]/_dt +
					innerProduct(bas_grad[i][l], bas_grad[j][l]));
				}
			}
		}
	}
};	
\end{lstlisting}
In above code, I add the new items into the element stiff matrix by adding them in the command:
\begin{lstlisting}
	elementMatrix(i,j) += Jxw*(bas_val[i][l]*bas_val[j][l]/_dt +
	innerProduct(bas_grad[i][l], bas_grad[j][l]));
\end{lstlisting}

Then I add the new item in the right hand side by this:
\begin{lstlisting}
Vector<double> rhs(fem_space.n_dof());
FEMSpace<double,DIM>::ElementIterator the_ele = fem_space.beginElement();
FEMSpace<double,DIM>::ElementIterator end_ele = fem_space.endElement();
for (;the_ele != end_ele;++ the_ele) {
	double vol = the_ele->templateElement().volume();
	const QuadratureInfo<DIM>& qi = the_ele->findQuadratureInfo(4);
	u_int n_q_pnt = qi.n_quadraturePoint();
	...
	...
	const std::vector<int>& ele_dof = the_ele->dof();
	u_int n_ele_dof = ele_dof.size();
	for (u_int l = 0;l < n_q_pnt;++ l) {
		double Jxw = vol*qi.weight(l)*jac[l];
		double f_val = _f_(q_pnt[l]);
		for (u_int i = 0;i < n_ele_dof;++ i) {
			// Build the rhs vector;
			rhs(ele_dof[i]) += Jxw*bas_val[i][l]*(u_h_val[l]/dt + f_val);
		}
	}
\end{lstlisting}
After these changes, I get the numerical result of $u$. But it shows like there exist a shake every 50 time steps iterations(the maximum value changes besides 1).\\	

Here are the graph:


%………………end……………
\newpage
\addcontentsline{toc}{chapter}{Bibliography}
\bibliographystyle{plain}
\bibliography{bibfile}

\end{document}